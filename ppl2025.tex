\documentclass[japanese]{jssst_ppl}
\usepackage{geometry}
\usepackage{amsmath,amssymb, amsfonts, latexsym, mathtools}
\usepackage{minted}
\usepackage[all]{xy}
\usepackage{amsthm}
\usepackage{stmaryrd}
\usepackage{natbib}
\usepackage[svgnames]{xcolor}
\usepackage{here}
\usepackage[inline]{enumitem}
\usepackage{url}
\usepackage{tikz}
\usepackage{bussproofs}
\usepackage{xhfill}
\usepackage{ulem}
\usetikzlibrary{calc,positioning,fit}

\definecolor{bg}{rgb}{.9, .9, .9}
\theoremstyle{definition}
\newtheorem{theorem}{定理}
\newtheorem*{theorem*}{定理}
\newtheorem{definition}[theorem]{定義}
\newtheorem*{definition*}{定義}
\newtheorem{example}[theorem]{例}
\newtheorem*{example*}{例}

\def\coqin#1{\text{\mintinline[escapeinside=||]{ssr}{#1}}}

\newcommand{\bind}{\rm{bind}}
\newcommand{\ret}{\rm{return}}
\newcommand{\binds}{ \gg = }

\title{遅延モナドを用いた一般再帰関数に対する等式変形による検証}
\author{川上 竜司, Jacques Garrigue, 才川 隆文}
\inst{ 名古屋大学多元数理学研究科\\
  \texttt{ \{ryuji.kawakami.c3, garrigue\}@math.nagoya-u.ac.jp, tscompor@gmail.com}}
\begin{document}
\maketitle
\begin{abstract}
  CoqのライブラリMonaeは、モナディック等式変形を用いてプログラムの計算効果に関する
  検証を可能にする。現在Monaeでは、状態モナドや確率モナドなど、様々なモナドを
  サポートすることで多様なプログラムを扱うことができるが、構造的でない再帰関数
  の扱いが難しい。一方で、余帰納的に定義される遅延モナド
  を用いると一般再帰関数を表現できることが知られている。
  本研究では、遅延モナドに対するwhileを用いた適切なインターフェイスを定義し、
  その健全性を形式的に証明することで、Monaeを用いた一般再帰関数に対する検証を可能にした。
  また、モナドトランスフォーマーを用いて他のモナドと組み合わせることで、計算効果を含みうるより
  一般的なプログラムに対するMonaeを用いた検証を可能にする。
\end{abstract}

%自分で定義した関数に対してはcoqinで書く
\section{初めに}
純粋関数型プログラムはその参照等価性としての性質から、等式変形による検証に適している。
さらには、モナドと呼ばれる構造を用いることで、計算効果を表すことができ、関数型プログラミング言語Haskell
をはじめとした様々な言語において採用されている。
Gibbonsらは、モナドの持つ代数的な性質に着目し、それぞれのモナドのインターフェイスを、等式の
集まりとして定義することで、計算効果の持つプログラムに関する等式変形による検証、モナディック等式変形を
提案した\cite{10.1145/2034574.2034777}。
%モナドトランスフォーマーもしくはtyped store monadについて書く?


\paragraph{Monae}
定理証明支援系Coq\cite{Coq}でモナディック等式変形を用いた検証を可能にするツールはMonae\cite{monae}である。
Monaeは、モナディック等式変形を行うための等式の集まりであるインターフェイスと、
その健全性を保証するモデルから構成される。
Coqを用いることで、検証の正しさを保証し、Coqの数学ライブラリMathComp/SSReflect\cite{Gonthier_Mahboubi_2010}
を用いた簡潔な証明が可能になる。
Monaeでは、モナディック等式変形で必要となるインターフェイスの階層を
Hierarchy Builder\cite{cohen:hal-02478907}を用いて実装することで、
複数のモナドの組み合わせや、再利用可能な構造的な証明を可能にする。
%現在monaeでできることについて書く?
\paragraph{構造的でない再帰関数の扱い}
定理証明支援系では無矛盾性の保証のため、停止しない関数を定義できない。
Coqの\coqin{Fixpoint}コマンドでは、引数の持つ整礎な順序関係が構文的に自明な構造的再帰関数については定義できる。
そうでないときは、\coqin{Equation}コマンドなどを用いて引数が整礎な順序関係を持ち減少する値であることの
証明とともに定義する必要がある。\\
例えば、複雑な再帰を行うマッカーシーの91関数はCoqでは直ちに定義することはできない。
\begin{minted}[bgcolor=bg,numbers=left,xleftmargin=1.5em]{ocaml}
let rec mc91 m = if 100 < m then m - 10 else mc91 (mc91 (m+11))
\end{minted}
さらには、停止性の未解決なコラッツ関数などは再帰関数として定義することはできない。
%停止性の証明なしに扱える さらに、構造
%一般再帰が扱える さらに、quick sortなども
%また、停止性の未解決なコラッツ関数などは定義することができない。
%interaction treeを全面的に出す?
\paragraph{余帰納型を用いた一般再帰関数の定義}
定理証明支援系で構造的でない再帰関数を扱う他の方法として、無限個のコンストラクターを持つデータを
定義する際に用いられる余帰納的定義を用いる手法がある。
Coqでは、ガード制約を満たす限り、無限にコンストラクターを適用することができるため、
それを用いて無条件な再帰呼び出しを行い、構造的でない再帰関数を扱うことができる。
それらの関数は、戻り値が余帰納型データとなるが、停止性の証明なしに定義することができる。\\
%例えば、\cite{10.1145/3371119}では、coinductive typeを用いて定義されたデータ構造ITreeを用いて
%インタラクションのある一般再帰関数を表現している。\\
Caprettaらの提案した遅延モナド\cite{lmcs:2265}を用いるとそういった余再帰的な関数呼び出し
を行うプログラムをモナディックプログラムとして表すことができる。
したがって、本研究では、遅延モナドに対する適切なインターフェイスを定義することで
Monaeを用いた停止性の保証できない一般再帰関数に対する検証を試みた。
%\subsection{interaction treeにおける構造的でない再帰関数の扱い}
\paragraph{完全エルゴートモナド}
遅延モナドのインターフェイスを定義するにあたって、完全エルゴートモナド\cite{ADAMEK20101306}
を参考にした。
完全エルゴートモナドは、代数的に再帰構造を扱うIteration theory\cite{1993Bloom}に
対応しており、イテレーションと呼ばれる、
各$f : X \rightarrow M (Y + X)$を$f^{\dagger} : X \rightarrow M Y$に対応させるオペレーター
$(\_)^{\dagger}$に関する4つの公理を満たすモナドとして定義される。\\
例えば、等式fixpointは次の可換図式で表される。
\xymatrixcolsep{2.5cm}
\[
  \xymatrix{
  X \ar[r]^{f^{\dagger}} \ar[d]^{f} & M A  \\
  M (A + X) \ar[r]^{M[\eta_A, f^{\dagger}]}  & M^2 A \ar[u]^{\mu_A}
  }
\]
この規則は関数fがAの値を返すまで繰り返し計算を行った結果が、$f^{\dagger}$に等しいことを表している。\\
遅延モナドは有限回の計算ステップを無視する同一視を行うことで
完全エルゴートモナドとなることが知られている\cite{10.1007/978-3-319-67729-3_3}。
%\subsection{Typed store monad transformer}

\paragraph{モナドトランスフォーマーを用いた計算効果の合成}
遅延モナドを用いることで、一般再帰関数を扱えるが純粋な関数に限られる。
例えば、参照とwhile文を用いて階乗を計算する\mintinline{ocaml}|factorial| は
遅延モナドだけでは表現できない。
\begin{minted}[bgcolor=bg,numbers=left,xleftmargin=1.5em]{ocaml}
let factorial n =
  let r = ref 1 in
  let l = ref n in
  while !l != 0 do
  r := !r * !l;
  l := !l - 1;
  done;
\end{minted}
そこで、本研究では、複数のモナドを変換により組み合わせることができる
モナドトランスフォーマーを用いて、例外モナドや、ocamlの参照を表現する
ために導入された型付きストアモナド\cite{affeldt2023practicalformalizationmonadicequational}と組み合わせることで、
複数の計算効果をもつプログラムに対するMonaeを用いた検証を可能にした。

\paragraph{本稿の貢献と構成}
本稿の貢献は以下のようにまとめられる。
\begin{itemize}
  \item 遅延モナドの計算的同値性に関する規則を含むインターフェイスを定義し、その健全性を
        形式的に示すことで、一般再帰関数に関する検証をMonaeで行うことを可能にした。
  \item モナドトランスフォーマーを用いた他のモナドとの組み合わせや一般項書き換え
        を可能にするsetoidライブラリを用いることで一般再帰関数に関するMonaeによる検証の有用性を高めた。
\end{itemize}
以下、2節で遅延モナドのインターフェイスの詳細ついて、3節で型付きストアモナドとの組み合わせについて、
説明する。
また4節で関連研究について、5節でまとめと課題について論じる。
なお、本研究のコードは以下のulrから確認することができる。\\
\begin{center}
  \url{https://github.com/Ryuji-Kawakami/monae/tree/delaypull}
\end{center}

\section{遅延モナドのMonaeにおける実装}
\subsection{遅延モナドの定義}
%Cofixpointの説明?
遅延モナドは、余帰納的に定義されることで
停止しない関数の計算を表現することが可能である。\\
遅延モナドを構成する関手Delayは、(A:Type) $\rightarrow$ (X = A + X \text{の最大不動点})
という型を持つ関数である。
Coqでは、\coqin{CoInductive}コマンドと、各最大不動点への埋め込みを表すコンストラクター\coqin{DNow}, \coqin{DLater}
を用いて定義することができる。
\begin{minted}[bgcolor=bg,numbers=left,xleftmargin=1.5em]{ssr}
CoInductive Delay (A : Type) : Type :=
  |DNow : A -> Delay A
  |DLater : Delay A -> Delay A.
\end{minted}
遅延モナドのreturnオペレーターは、コンストラクタ\coqin{DNow}であり、
bindオペレーターは\coqin{CoFixpoint}コマンドを用いて定義される。
\begin{minted}[bgcolor=bg,numbers=left,xleftmargin=1.5em]{ssr}
Let ret (a:A) := DNow a
CoFixpoint bind (m: Delay A) (f: A -> Delay B ) :=
  match m with
  |DNow a => f a
  |DLater d => DLater (bind d f)
  end.
\end{minted}
さて、遅延モナドに付随するオペレーターとして、繰り返し処理を行うwhileを定義する。
whileオペレーターは完全エルゴートモナドの$(\_)^{\dagger}$オペレーターに相当する。\\
\coqin{CoFixpoint}コマンドを用いて定義され、右埋め込みの値\mintinline{ssr}|inr a|が値a
での繰り返しの継続、左埋め込みの値\mintinline{ssr}|inl b|が値bを戻り値とする繰り返し
の終了を表す。
\begin{minted}[bgcolor=bg,numbers=left,xleftmargin=1.5em]{ssr}
CoFixpoint while {A B} (body: A -> M (B + A)) : A -> M B :=
  fun a => (body a) >>= (fun ab => match ab with
                                   |inr a => DLater (while body a)
                                   |inl b => DNow b end).
\end{minted}
また、whileオペレーターを用いて不動点オペレーターの定義が行えることが知られている。
例えば、Filinskiは第一級継続を用いて、whileオペレーターから
不動点オペレーターを定義している\cite{recursion_from_iteration}。


%CoFixpointに関する説明を詳しく?
\subsection{計算的同値性}
さて、ここで等式変形による検証は、遅延モナドを用いて表したプログラムに対しては適さない。
例えば、階乗を計算する関数\coqin{fact}を遅延モナドを用いて定義した場合、\coqin{fact 3}は
明示的な計算ステップ\coqin{DLater}を3つ含むため、\coqin{DNow 6}と一致しないためである。
したがって、計算的な同値性を表す関係が必要である。
そこで、有限個の\coqin{DLater}を除いて等しい場合、またはどちらも\coqin{DLater}が無限個続く場合に計算的に
等しいとみなす関係\coqin{wBisim}を\cite{lmcs:2265}と同様に次のように導入した。\\
まず、計算がある値で停止する性質を\coqin{Terminate}という帰納的な関係で定義する。\\
\mintinline{ssr}|Terminate d a| とは、dが計算の結果、値aを返すことである。
\begin{minted}[bgcolor=bg,numbers=left,xleftmargin=1.5em]{ssr}
Inductive Terminates A : Delay A -> A -> Prop :=
  |TDNow a : Terminates (DNow a) a
  |TDLater d a : Terminates d a -> Terminates (DLater d) a.
\end{minted}
次に、この述語を用いて、\coqin{wBisim}を余帰納的関係として定義する。
つまり、d1,d2が有限個の\coqin{DLater}を除いて等しい場合\coqin{wBTerminate d1 d2 a}が成り立ち、
d1,d2がどちらも\coqin{DLater}が無限個続く場合は、\coqin{wBLater}により
余帰納法を用いることで、\coqin{wBisim d1 d2}を示すことができる。
\begin{minted}[bgcolor=bg,numbers=left,xleftmargin=1.5em]{ssr}
CoInductive wBisim A : Delay A -> Delay A -> Prop :=
  |wBTerminate d1 d2 a :
    Terminates d1 a -> Terminates d2 a -> wBisim d1 d2
  |wBLater d1 d2 : wBisim d1 d2 -> wBisim (DLater d1) (DLater d2).
\end{minted}

\subsection{余帰納法を用いた等式の証明}
%coinductionとcoqのcofixがどう対応しているか
%coqではcofixは、inductionがfixで証明項を作るのと同様に、cofixで証明項を作ることで
%coinductionを実装している。つまり、この例では cofix CIH (d : M A) : Oeq (DLater d) d :=
%から始まる証明項を作ることで、余帰納法を行うことができる。
%さて、cofixの型規則は以下である。(論文引用しながら)
%論文の型規則、ガーデッドコンディションを日本語で書き下す、簡単に
%The predicate Co ensures that, going downwards in M, after a certain number of abstractions and case analysis, there is always a constructor. Once a constructor has been placed, the predicate C1 allows to use f in its recursive arguments. 
%(1)余帰納呼び出しは、CoInductiveのコンストラクタを含まなければならない

%(2)余帰納呼び出しは、CoInductiveのコンストラクタで包まなければならない

%(3)余再帰呼び出しは、CoInductiveのコンストラクタの引数の一番外側で行わなければならない
%つまり、
%その上で、この証明は次のように行った。証明中のcofixで、、、証明の中では、、、からguarded conditionを
%満たしていることがわかる。
%each recursive call in the definition must be protected by at least one constructor, and only by constructors.


ここでは、\coqin{wBisim_DLater}を例にどのように余帰納法の原理を用いて、
\coqin{wBisim}に関する性質をCoq上で示したかを説明する。

\begin{minted}[bgcolor=bg,numbers=left,xleftmargin=1.5em]{ssr}
  Lemma wBisim_DLater A : forall (d : M A), wBisim (DLater d) d.
  \end{minted}

Coqでは、帰納法で\coqin{fix}オペレーターを用いて証明項を構成するのと同様に、
余帰納法では\coqin{cofix}オペレータを用いて証明項を構成する。
つまり、この例では、\\
\coqin{cofix CIH (d : M A) : Oeq (DLater d) d := |$\cdots$|}
から始まる項を構成すればよい。
さて、\coqin{cofix}に関する型規則は以下である\cite{10.1007/3-540-60579-7_3}。
\begin{prooftree}
  \AxiomC{$\Gamma \vdash B:Set$}
  \AxiomC{$\Gamma, f:B \vdash N:B $}
  \AxiomC{$\mathcal{C}\{ f, N \}$}
  \RightLabel{\text{CoFix}}
  \TrinaryInfC{$\Gamma \vdash \text{CoFix} \, f:B := N:B$}
\end{prooftree}

ただし、CoIを余帰納的に定義された型、PをCoIのパラメーター、$ B := \Pi_{z:Z} (\text{CoI} \, P \, (z))$
$\mathcal{C}\{ f, N \}$を$f$と$N$に関する条件とする。この条件は構文的ガード制約と呼ばれ、
$N$に関して帰納的に定義される。特に次のような性質を満たす
\begin{itemize}
  \item $N$はコンストラクターを含まなければならない。
  \item 余再帰呼び出し$f$はコンストラクターの内側で行われ、その外側では、場合わけとラムダ抽象しか行われていない。
\end{itemize}
さて、このことに注意して、次のように証明を行った。\\
cofixタクティックで、余再帰呼び出しである\coqin{apply CIH}は、コンストラクター\coqin{wBLater}
の内側で行われており、その前には\coqin{d}に関する場合わけしか行っていないため、確かに構文的ガード
制約を満たす。

\begin{minted}[bgcolor=bg,numbers=left,xleftmargin=1.5em]{ssr}
  Lemma wBisim_DLater A : forall (d : M A), wBisim (DLater d) d.
  Proof.
  cofix CIH => d.
  case: d => [a|d'].
  - apply : wBTerminate.
    + by apply/TDLater/TDNow.
    + by apply TDNow.
  - apply wBLater.
    by apply CIH.
  Qed.
  \end{minted}

\subsection{遅延モナドのインターフェイス}
以上のことを踏まえて、遅延モナドのインターフェイスを表1,2,3のように定義した。\\
まず、遅延モナドのオペレイターは、繰り返し処理を行うwhileである。また、計算の等さを表す関係である、
\coqin{wBisim} \, $ \approx $を導入した。
表2は\coqin{wBisim}に関する規則である。
前半の3つの規則は、wBisimが同値関係であることを表している。
後半の3つの規則は、検証上有用であると考え導入した。
bindmwBはbindの引数が同じ計算結果を表しているのならば、fに渡した結果も同じ計算結果になることを表している。\\
bindfwBは同じfとgが同じ計算をするならば、bindで同じ引数を渡した結果も同じ計算結果になることを表す。\\
whilewBは、while文でその都度繰り返す処理が、計算的に等しいならば、
while文全体として等しいことを表す。\\
後半の3つの規則は、whileとbindが同値類を保つことを表している。
したがって、パラメトリックモルフィズムとして登録することで、等式変形に近い証明を可能にした。
詳細については2.5節で説明する。
\\
表3はwhileオペレーターに関する規則である。
完全エルゴートモナドの定義を参考にした。\\
fixpointEは、whileによって、繰り返す処理をすることができることを表す規則である。\\
naturalityEは、while文により行った計算結果を、次の処理に渡す場合、
それを一つのwhile文により記述できることを表す。\\
codiagonalEは、連続して入れ子になったwhile文を一つのwhile文にすることができることを表す。\\
これらの規則を用いた等式変形では、モデルにおいて、健全性を示した時のような余帰納法を用いた証明
をする必要がなく、簡潔で直感的な検証が可能になる。

\begin{table}[H]
  \caption{オペレーターと関係}
  \centering
  \begin{tabular}{|c|}
    \hline
    \coqin{while :  (A -> M(B + A)) -> A -> M B} \\
    \coqin{wBisim :  M A -> M A -> Prop}         \\
    \hline
  \end{tabular}
\end{table}

\begin{table}[H]
  \caption{\coqin{wBisim}に関する規則}
  \centering
  \begin{tabular}{|l|l|}
    \hline
    \coqin{wBisim_refl}  & \coqin{ a |\scalebox{0.8}{$\approx$}| a}                                                                       \\
    \coqin{wBisim_sym}   & \coqin{ a |\scalebox{0.8}{$\approx$}| b -> b |\scalebox{0.8}{$\approx$}| a}                                    \\
    \coqin{wBisim_trans} & \coqin{ a |\scalebox{0.8}{$\approx$}| b -> b |\scalebox{0.8}{$\approx$}| c -> a |\scalebox{0.8}{$\approx$}| c} \\
    \coqin{bindmwB}      & \coqin{d1 |\scalebox{0.8}{$\approx$}| d2 -> d1 >>= f |\scalebox{0.8}{$\approx$}| d2>>= f}                      \\
    \coqin{bindfwB}      & \coqin{(forall a, f a |\scalebox{0.8}{$\approx$}| g a) -> d >>= f |\scalebox{0.8}{$\approx$}| d >>= g}         \\
    \coqin{whilewB}      & \coqin{(forall a, f a |\scalebox{0.8}{$\approx$}| g a) -> while f a |\scalebox{0.8}{$\approx$}| while g a}     \\
    \hline
  \end{tabular}
\end{table}

\begin{table}[H]
  \caption{whileに関する規則}
  \centering
  \begin{tabular}{|l|l|}
    \hline
    \coqin{fixpointE }   & \coqin{while f a |\scalebox{0.8}{$\approx$}| (f a) >>= (sum_rect Ret (while f)) }   \\
    \coqin{naturalityE } & \coqin{while f a >>= g |\scalebox{0.8}{$\approx$}|}                                 \\
                         & \coqin{while (fun y => (f y) >>=}                                                   \\
                         & \coqin{  (sum_rect (M |\#| inl o g) (M |\#| inr o Ret))) a}                         \\
    \coqin{codiagonalE } & \coqin{while ((M |\#| ((sum_rect  idfun inr))) o f ) a |\scalebox{0.8}{$\approx$}|} \\
                         & \coqin{  while (while f) a}                                                         \\
    \hline
  \end{tabular}
\end{table}

\subsection{一般書き換え}
ユーザーが独自に定義したパラメトリック同値関係に対するrewriteタクティックスの使用を
Setoidライブラリは可能にする。
一般再帰関数に対する検証では、等式変形による検証ではなく、計算的な同値性である\coqin{wBisim}に
関する等価性変形を行う必要がある。\\
したがって、\coqin{wBisim}をパラメトリック同値関係としてインスタンス化した。
\begin{minted}[escapeinside=||,bgcolor=bg,numbers=left,xleftmargin=1.5em]{ssr}
Add Parametric Relation A : (M A) (@wBisim M A)
  reflexivity proved by (@wBisim_refl M A)
  symmetry proved by (@wBisim_sym M A)
  transitivity proved by (@wBisim_trans M A) as wBisim_rel.
        \end{minted}
インターフェイスの規則であるbindmwBとbindfwB、whilewBがそれぞれ
関数bind、whileが同値関係\coqin{wBisim}を保つことを表している。
したがって、bind、whileを Parametric morphismとしてインスタンス化した。
定義の中に現れる\\
任意の値に対して関数が同じ同値類の値を返すことを表す\\
\coqin{(pointwise_relation A (@wBisim M B))}を用いて関数の間の関係を定義する。
これにより、図1,2のような書き換えが可能になる。
さらに、\coqin{setoid_rewrite}を用いることで、束縛変数
を含む項に関する書き換えが可能である。このこととついては、5.3節で具体的に説明する。

\begin{minted}[escapeinside=||,bgcolor=bg,numbers=left,xleftmargin=1.5em]{ssr}
Add Parametric Morphism A B : bind
  with signature (@wBisim M A) ==>
    (pointwise_relation A (@wBisim M B)) ==> (@wBisim M B) as bindmor.

  Add Parametric Morphism A B : while
  with signature (pointwise_relation A (@wBisim M (B + A))) ==>
    @eq A ==> (@wBisim M B ) as whilemor.
  \end{minted}
これにより、Monaeを用いた他の検証と同様なrewriteタクティックスを中心とした証明が可能になった。
\begin{figure}[h]
  \centering
  \begin{minipage}{0.38\textwidth}
    \begin{minted}[escapeinside=||]{ssr}
 H1 : d1 |$\approx$| d2
 H2 : forall a, f |$\approx$| g
| \xdotfill{1pt}[black] |
 d1 >>= f |$\approx$ \, $\cdots$|
  \end{minted}
  \end{minipage}
  \begin{minipage}{0.2\textwidth}
    \begin{center}
      \scalebox{1.1}{$\xRightarrow[\coqin{rewrite H1 H2.}]{}$}
    \end{center}
  \end{minipage}
  \begin{minipage}{0.38\textwidth}
    \begin{minted}[escapeinside=||]{ssr}
H1 : d1 |$\approx$| d2
H2 : forall a, f a |$\approx$| g a
|\xdotfill{1pt}[black] |
d2 >>= g |$\approx$ \, $\cdots$|
  \end{minted}
  \end{minipage}
  \caption{rewrite with bind}
\end{figure}

\begin{figure}[h]
  \centering
  \begin{minipage}{0.38\textwidth}
    \begin{minted}[escapeinside=||]{ssr}
 H : forall a, f a |$\approx$| g a.
| \xdotfill{1pt}[black] |
 while f b |$\approx$ \, $\cdots$|
  \end{minted}
  \end{minipage}
  \begin{minipage}{0.2\textwidth}
    \begin{center}
      \scalebox{1.1}{$\xRightarrow[\coqin{rewrite H.  }]{}$}
    \end{center}
  \end{minipage}
  \begin{minipage}{0.38\textwidth}
    \begin{minted}[escapeinside=||]{ssr}
H : forall a, f a |$\approx$| g a.
|\xdotfill{1pt}[black] |
while g b |$\approx$ \, $\cdots$|
  \end{minted}
  \end{minipage}
  \caption{rewrite with while}
\end{figure}

\iffalse
  この内容はappendixに入れる?
  \subsection{Monadの満たす等式}
  モナドは、bindとreturnを持ち、表4の等式を満たす型クラスとして特徴づけられる。
  \begin{table}[b]
    \caption{The laws of the monad}
    \centering
    \begin{tabular}{|l|l|}
      \hline
      \coqin{bindretf} & \coqin{Ret a >>= f = f a}                            \\
      \coqin{bindmret} & \coqin{m >>= Ret = Ret}                              \\
      \coqin{bindA}    & \coqin{(m >>= f) >>= g = m >>= (fun x => f x >>= g)} \\
      \hline
    \end{tabular}
  \end{table}


  Monaeでは、各Monadをインスタンス化した際にこれらの等式と関連する補題を用いることができるようになる。
  特にここでは、表4の等式を用いて検証する。

\fi

\subsection{マッカーシーの91関数に対する検証}

遅延モナドのインターフェイスを用いた検証例として、再帰的に定義される関数であり、
$m \leq 101$の時、必ず91を返すという性質をもつマッカーシーの91関数\coqin{mc91}を扱う。

\begin{minted}[bgcolor=bg,numbers=left,xleftmargin=1.5em]{ocaml}
let rec mc91 m = if 100 < m then m - 10 else mc91 (mc91 (m+11))
   \end{minted}

$m \leq 100$の時、11足した値で二重に再帰するため、構造的ではない再帰関数である。\\
したがって、mc91関数をdelayモナドを用いて表し、自然数$m \leq 100$について
91が戻り値となることを示した。\\
まず、\coqin{while}オペレーターを用いてmc91関数を用いて表す。nが残っている再帰の深さ、mが計算している値である。

\begin{minted}[bgcolor=bg,numbers=left,xleftmargin=1.5em]{ssr}
Let mc91_body nm :=
    match nm with (n, m) =>
    if n==0 then ret (inl m)
            else if m > 100
                 then ret (inr(n.-1, m - 10))
                 else ret (inr(n.+1, m + 11))
    end.
Let mc91 n m := while mc91_body (n.+1, m).
    \end{minted}

mc91関数が91を返すことを示す際、本質的な性質は、次の補題\coqin{mc91succE}である。
この補題と、\coqin{mc91 n 101 |$\approx$| Ret 91}であることと、$ k =  90 - m $に関する帰納法により従う。

\begin{minted}[bgcolor=bg,numbers=left,xleftmargin=1.5em, escapeinside=||]{ssr}
Lemma mc91succE n m : 90 <= m < 101 -> mc91 n m |\scalebox{0.8}{$\approx$}| mc91 n (m.+1).
\end{minted}

さて、この補題を図3のように示した。

\begin{figure}[H]
  \centering
  \begin{minted}[escapeinside=||,mathescape = true, frame = single]{ssr}
mc91 n m
 |$\llbracket$| rewrite /mc91. (*definition of mc91*) |$\rrbracket$|
 | \scalebox{0.8}{$\approx$} | while mc91_body (n.+1, m)
 |$\llbracket$| rewrite fixpointE. |$\rrbracket$|
 | \scalebox{0.8}{$\approx$} | (if 100 < m then Ret (inr (n, m - 10))
              else Ret (inr (n.+2, m + 11))) >>=
                   sum_rect Ret (while mc91_body)
 |$\llbracket$| rewrite ifN //. (* $ m \leq 100$ *)|$\rrbracket$|
  |\scalebox{0.8}{$\approx$}| Ret (inr (n.+2, m + 11)) >>= sum_rect Ret (while mc91_body)
 |$\llbracket$| rewrite bindretf /= fixpointE /=. |$\rrbracket$|
  |\scalebox{0.8}{$\approx$}| while mc91_body (n.+2, m + 1)
 |$\llbracket$| rewrite bindretf fixpointE /=. |$\rrbracket$|
  |\scalebox{0.8}{$\approx$}| (if 100 < m + 11
      then Ret (inr (n.+1, m + 11 - 10))
      else Ret (inr (n.+3, m + 11 + 11))) >>=
          sum_rect Ret (while mc91_body)
  |$\llbracket$| rewrite ltn_add2r ifT //. (* $90 \leq m \implies 100 < m + 11 $ *) |$\rrbracket$|
  Ret (inr (n.+1, m + 11 - 10)) >>= sum_rect Ret (while mc91_body)
 |$\llbracket$| rewrite  bindretf fixpointE /= fixpointE. |$\rrbracket$|
  |\scalebox{0.8}{$\approx$}|  while mc91_body (n.+1, m + 11 - 10) = mc91 n (m+1)
    \end{minted}
  \caption{mc91関数に対する証明 }
\end{figure}

\section{型付きストアモナドとの組み合わせ}

遅延モナドのインターフェイスを定義することで、一般再帰関数に対する検証が可能になった。
この手法をより多くの関数へ適用するためには、遅延モナドを他のモナドと組み合わせることにより
複雑な副作用を表現する必要がある。そこで、\cite{practicalformalizationmonadicequational}
において導入された型付きストアモナドとの組み合わせた遅延型付きストアモナドを定義した。

\subsection{型付きストアモナド}
Coqgen\cite{ValidatingOS}により変換したコードの、
参照を表すために導入されたモナドが型付きストアモナドである。\\
型と値のレコードbindingのリストを状態として使うことで、型付きストアを表している。\\
また参照を扱うためのオペレーターである\coqin{cnew}、 \coqin{cget}、 \coqin{cput}を持つ。

\begin{minted}[bgcolor=bg,numbers=left,xleftmargin=1.5em]{ssr}
Record binding :=
  mkbind { bind_type : ml_type; bind_val : coq_type bind_type }.
   \end{minted}
本研究では、型付きストアモナドを、例外モナドトランスフォーマー\coqin{MX}と
状態モナドトランスフォーマー\coqin{MS}の合成で定義された型付きストアモナドトランスフォーマーに
変更し、それぞれのモナドトランスフォーマーが遅延モナドの構造を保つことを示すことで、
遅延型付きモナドを定義した。
\begin{center}

  \coqin{MS (seq binding) option_monad} \ $\Rightarrow$ \
  \coqin{MS (seq binding) (MX unit M)}
\end{center}



\subsection{状態モナドトランスフォーマーとの組み合わせ}
状態モナドトランスフォーマーstateTは次のように定義される。\\
Sが状態、Mが変換前の関手、として\coqin{MS},\coqin{retS},\coqin{bindS}がそれぞれ変換後の関手、return, bindである。

\begin{minted}[bgcolor=bg,numbers=left,xleftmargin=1.5em]{ssr}
Let MS := fun A => S -> M (A * S).
Let retS := fun A => curry Ret.
Let bindS m f := fun s => m s >>= uncurry f.
Let liftS m := fun s => m >>= (fun x => Ret (x, s)).
        \end{minted}
さて、stateTにより状態モナドと組み合わせるためには、遅延モナドがstateTで変換後、遅延モナドであることを示す
必要がある。そこで、\cite{PIROG2013309}を参考に次のように\coqin{whileDS}と\coqin{wBisimDS}を定義した。
関数\coqin{dist1}は、分配法則を表し、型を合わせるために定義した。
\iffalse
  そこで、次の定理を用いた。\\

  %monad for behaviorの定理をかく
  %ここもなくてもよい。参考にした論文のい挙げる
  随伴にcurryとuncurryを

  特に状態モナドが随伴$(- \otimes A) \vdash (A \rightarrow - )$から導出されるため次のように
  while operatorとwBisimを定義した。関数dist1は、分配法則を表し、型を合わせるために定義した。
\fi
\begin{minted}[bgcolor=bg,numbers=left,xleftmargin=1.5em]{ssr}
M:delayMonad
Let DS := MS M
Let whileDS (body : X -> DS (Y + X)) :=
  curry (while (M # dist1 o uncurry body)).

Let wBisimDS (ds1 ds2 : DS A) : Prop :=
  forall s : S, wBisim (ds1 s) (ds2 s).
\end{minted}

\subsubsection{例外モナドトランスフォーマーとの組み合わせ}
例外モナドトランスフォーマーexceptTは次のように定義される。\\
Zが例外の集合、Mが変換前の関手、として\coqin{MX},\coqin{retX},\coqin{bindX}がそれぞれ変換後の関手、return, bindである。

\begin{minted}[bgcolor=bg,numbers=left,xleftmargin=1.5em]{ssr}
Let MX := fun X => M (Z + X).
Let retX : idfun --> MX := fun X x => Ret (inr x).
Let bindX t f :=
  t >>= fun c => match c with inl z => Ret (inl z) | inr x => f x end.
        \end{minted}
stateTの場合と同様に、遅延モナドがexceptTで変換後遅延モナドであることを示す必要がある。
そこで、次のように定義した。
関数\coqin{DEA}を合成することでにより、エラーが発生した際\coqin{inl (inl u)}を返すことで、繰り返しを終了する。

\begin{minted}[bgcolor=bg,numbers=left,xleftmargin=1.5em]{ssr}
M: delayMonad
Let DE := MX unit M.
Let whileDE (body : A -> DE (B + A)) : DE B := while (DEA o body)
Let wBisimDE (d1 d2 : DE A) := wBisim d1 d2.
        \end{minted}

\subsection{検証の具合例}
\coqin{M : delay_typed_storemonad}を用いることで、参照に関する
計算とwhileを用いた繰り返しを含むプログラムを表現可能である。\\
例えば、参照と繰り返しを用いて階乗を計算するプログラム\coqin{factdts}を以下のように定義できる。
\begin{minted}[bgcolor=bg,numbers=left,xleftmargin=1.5em]{ssr}
Let factdts_aux_body r n : M (unit + nat) :=
  do v <- cget r;
     match n with
     |O => do _ <- cput r v; Ret (inl tt)
     |S m => do _ <- cput r (n * v); Ret (inr m)
     end.
Let factdts_aux n r := while (factdts_aux_body r) n.
Let factdts n := do r <- cnew ml_int 1;
                 do _ <- factdts_aux n r ;
                 do v <- cget r; Ret v.
            \end{minted}
\coqin{factdts}について、実際に階乗を計算していること
を検証する。つまり次\coqin{factn}と計算として一致することを示した。

\begin{minted}[bgcolor=bg,numbers=left,xleftmargin=1.5em]{ssr}
Fixpoint fact n := match n with |O => 1 |m.+1 => n * fact m end.
Let factn n := do r <- cnew ml_int (fact n);
               do v <- cget r; @ret M _ v.
\end{minted}

ここでは、\coqin{factdts}のwhileオペレーターを用いた部分\coqin{fact_aux}が\coqin{fact}を用いた形に書き換えられることについて
取り上げる。この書き換えと、型付きストアモナドの等式を用いることで、\coqin{factn}に一致することが従う。
証明はnに関する帰納法により図4,5のように行った。\\
\coqin{under}タクティックを用いることで、bindに関する束縛変数を含む項に関する等式変形
を行っている。

\begin{figure}[H]
  \centering
  \begin{minted}[escapeinside=~~, frame = single]{ssr}
while (factdts_aux_body r) 0
~$\llbracket$~ rewrite fixpointE/= !bindA. ~$\rrbracket$~
~ \scalebox{0.8}{$\approx$} ~ cget r >>= (fun s => (cput r s >> Ret (inl tt)) >>=
                     sum_rect Ret (while (factdts_aux_body r)))
~$\llbracket$~ under eq_bind => s. (* rewrite under binder *) ~$\rrbracket$~
'Under[ (cput r s >> Ret (inl tt)) >>=
         sum_rect Ret (while (factdts_aux_body r)) ]
~$\llbracket$~ rewrite bindA bindretf/= -{1}(mul1n s). ~$\rrbracket$~ 
'Under[ cput r (1 * s) >> Ret tt ]
~$\llbracket$~ over. ~$\rrbracket$~
cget r >>= (fun s => cput r (1 * s) >> Ret tt)
= cget r >>= (fun s => cput r (fact 0 * s) >> Ret tt)
    \end{minted}
  \caption{n = 0の場合}
\end{figure}

n = n' + 1の時の証明の際、\coqin{setoid_rewrite IH}の部分では、
束縛変数を含む項に対する\coqin{wBisim}に関する等価性変形を行っている。これは、
第2.5節で説明したように、bindをパラメトリックモルフィズムとして、インスタンス化
しているため行うことができる。

\begin{figure}[H]
  \centering
  \begin{minted}[escapeinside=~~, frame = single]{ssr}
while (factdts_aux_body r) n'.+1
~$\llbracket$~ rewrite fixpointE/= !bindA. ~$\rrbracket$~
~ \scalebox{0.8}{$\approx$} ~cget r >>= (fun s => (cput r (n'.+1 * s) >> Ret (inr n')) >>=
                     sum_rect Ret (while (factdts_aux_body r)))
~$\llbracket$~ under eq_bind => s (* rewrite under binder *) ~$\rrbracket$~
'Under[ (cput r (n'.+1 * s) >> Ret (inr n')) >>=
     sum_rect Ret (while (factdts_aux_body r)) ]
~$\llbracket$~ rewrite bindA bindretf/=. ~$\rrbracket$~
'Under[ cput r (n'.+1 * s) >> while (factdts_aux_body r) n' ]
~$\llbracket$~ over. (* over *) ~$\rrbracket$~
cget r >>= (fun s => cput r (n'.+1 * s) >>
           while (factdts_aux_body r) n')
~$\llbracket$~ setoid_rewrite  IH (*rewrite using induction hypothese IH*)~$\rrbracket$~
~ \scalebox{0.8}{$\approx$} ~ cget r >>= (fun s => cput r (n'.+1 * s) >>
           (cget r >>= (fun s0 => cput r (fact n' * s0) >> Ret tt)))
~$\llbracket$~ under eq_bind => s (* rewrite under binder *) ~$\rrbracket$~
'Under[ cput r (n'.+1 * s) >>
       (cget r >>= (fun s0 => cput r (fact n' * s0) >> Ret tt)) ]
~$\llbracket$~ rewrite cputget -bindA cputput. (*laws for cput and cget *) ~$\rrbracket$~
  'Under[ cput r (fact n' * (n'.+1 * x)) >> Ret tt ]
~$\llbracket$~ over. (* over *) ~$\rrbracket$~
cget r >>= (fun s => cput r (n'.+1 * fact n' * s) >> Ret tt)
= cget r >>= (fun s => cput r (fact n'.+1 * s) >> Ret tt)
       \end{minted}
  \caption{n =  n' + 1の場合}
\end{figure}
\iffalse
  その後、その書き換えとtypedstoremonadの書き換え規則を用いることで、factnに一致することを示した。

  \begin{minted}[escapeinside=||]{ssr}
  Lemma factE n : factdts n |\scalebox{0.8}{$\approx$}| factn n.

  cnew ml_int 1 >>= (fun r : loc ml_int => factdts_aux n r >> (cget r >>= [eta Ret]))
  |《 rewrite under cnew mlint 1 >>= by bindfwB 》|

  factdts_aux n a >> (cget a >>= [eta Ret])
  |《 fixpointE 》|

  |《 rewrite under cnew mlint 1 >>= by bindfwB》|
  factdts_aux n a >> (cget a >>= [eta Ret])
  |《 fixpointE 》|

  cget a >>=
  (fun x : coq_type N ml_int => (cput a (fact n * x) >> Ret tt) >> (cget a >>= [eta Ret]))
  |《 rewrite under cget a >>= by eqbind》|
  (cput a (fact n * x) >> Ret tt) >> (cget a >>= [eta Ret])
  |《 fixpointE 》|
  cput a (fact n * x) >> (cget a >>= [eta Ret])
  |《 over 》|
  cnew ml_int 1 >>=
  (fun a : loc ml_int =>
   cget a >>=
   (fun x : coq_type N ml_int => cput a (fact n * x) >> (cget a >>= [eta Ret])))
   |《 fixpointE 》|
   cnew ml_int (fact n) >>= (fun r : loc ml_int => cget r >>= [eta Ret])

\end{minted}
\fi
\section{関連研究}
\subsection{一般再帰関数に対するCoq上での検証}
Xiaらは、余帰納的型を用いて定義されたデータ構造ITreeを用いて
インタラクションのある一般再帰関数を表現している\cite{10.1145/3371119}。この手法では、それぞれの副作用を
イベントとしてITreeに埋め込み、ITreeの間の弱双模倣性を満たす等価関係を定義することでプログラムの検証を可能にしている。
またそれに関連してZakowskiらは
ライブラリpakoを拡張したgpacoを用いることで弱双模倣性に関する等式論理をCoqで実装している\cite{10.1145/3372885.3373813}。
% Dijkstra Monads Forever: Termination
%free monadを使った手法

\subsection{Elgotモナドに関する理論}
完全エルゴートモナドに関する理論的研究としては、SergeyとPirógらが、Compete elogit monadの構造が
余帰納的一般再開モナドトランスフォーマーにより保たれることを示している\cite{GONCHAROV2015183}。
また、Simpsonらは、はiteration theoryの公理が完全であることを示している\cite{cacmfpo}。

\section{まとめと今後の課題}
本研究により、遅延モナドを通じてMonaeで一般再帰関数を扱うことができるようになった。
次に、モナドトランスフォーマーを用いて、状態変化や例外処理といった計算効果を含む一般再帰関数
をMonaeで扱うことができるようになった。
最後にSetoidライブラリを用いることで、等価関係に基づく変形による検証を
rewrite タクティクスを中心とした証明により行うことができた。\\
今後の課題としては以下である。
%let rec f p =を基準にしたインターフェイスの定義については触れる?
\paragraph{プログラムの部分正当性の検証}
現在、プログラムの実行後の状態について検証するには、factauxEの証明で行ったように
整礎関係を見つけ出し、その関係に関する帰納法を行う必要がある。
つまり、実質的にプログラムの停止性を示す必要があり、ホーア論理で行うようなループ不変条件を用いた
検証を行うことができない。したがって、ループ不変条件を用いた部分正当性の検証を
どのように等式変形の枠組みで導入するかが課題である。\\

また、今後の展望としては以下である。
\paragraph{確率モナドとの組み合わせ}
今回の研究では、モナドトランスフォーマーによるモナドの組み合わせを行った。
一方、確率モナドはモナドトランスフォーマーによる組み合わせが難しいことが知られている。%文献は?
Monaeでは、インターフェイスに異なるモナドのオペレーターとその間の等式を
追加することで、確率モナドと非決定性モナドの組み合わせを行っている。遅延モナドに対して
も同様の手法で組み合わせることで多様な検証が可能になると考えている。
\paragraph{一般的な双模倣性に関するMonaeでの検証}
今回の研究で用いた計算的同値性は、Delay A上の弱双模倣性を満たす関係である。
一方、一般的な双模倣性に関する性質を持つモナドのインターフェイスを定義し、
その健全性を示すことで、並行プログラムやStreamなどの無限の長さも持ちうる
データを扱うプログラムに関する検証が期待できる。


\bibliographystyle{alpha}
\bibliography{reference.bib}
\end{document}