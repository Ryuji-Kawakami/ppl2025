\documentclass[japanese]{jssst_ppl}
\usepackage{geometry}
\usepackage{amsmath,amssymb, amsfonts, latexsym, mathtools}
\usepackage{minted}
\usepackage[all]{xy}
\usepackage{amsthm}
\usepackage{stmaryrd}
\usepackage{natbib}
\usepackage[svgnames]{xcolor}
\usepackage{here}
\usepackage[inline]{enumitem}
\usepackage{url}
\usepackage{tikz}
\usepackage{bussproofs}
\usepackage{xhfill}
\usepackage{ulem}
\usetikzlibrary{calc,positioning,fit}

\definecolor{bg}{rgb}{.9, .9, .9}
\theoremstyle{definition}
\newtheorem{theorem}{定理}
\newtheorem*{theorem*}{定理}
\newtheorem{definition}[theorem]{定義}
\newtheorem*{definition*}{定義}
\newtheorem{example}[theorem]{例}
\newtheorem*{example*}{例}

\def\coqin#1{\text{\mintinline[escapeinside=||]{ssr}{#1}}}

\newcommand{\bind}{\rm{bind}}
\newcommand{\ret}{\rm{return}}
\newcommand{\binds}{ \gg = }

\title{Delayモナドを用いた一般再帰関数に対する等式変形による検証}
\author{川上 竜司, Jacques Garrigue, 才川 隆文}
\inst{ 名古屋大学多元数理学研究科\\
  \texttt{ \{ryuji.kawakami.c3, garrigue\}@math.nagoya-u.ac.jp, tscompor@gmail.com}}
\begin{document}
\maketitle
\begin{abstract}
  CoqのライブラリMonaeは、モナディック等式変形を用いてプログラムの計算効果に関する
  検証を可能にする。現在Monaeでは、状態モナドや確率モナドなど、様々なモナドを
  サポートすることで多様なプログラムを扱うことができるが、構造的でない再帰関数
  の扱いが難しい。一方で、余帰納的に定義されるDelayモナド
  を用いると一般再帰関数を表現できることが知られている。
  本研究では、Delayモナドに対するwhileを用いた適切なインターフェイスを定義し、
  その健全性を形式的に証明することで、Monaeを用いた一般再帰関数に対する検証を可能にした。
  また、モナドトランスフォーマーを用いて他のモナドと組み合わせることで、計算効果を含みうるより
  一般的なプログラムに対するMonaeを用いた検証を可能にする。
\end{abstract}

\section{初めに}
純粋関数型プログラムはその参照等価性としての性質から、等式変形による検証に適している。
さらには、モナドと呼ばれる構造を用いることで、計算効果を表すことができ、Haskellをはじめとした
様々な関数型プログラミング言語において採用されている。
Gibbonsらは、モナドの持つ代数的な性質に着目し、それぞれのモナドのインターフェイスを、等式の
集まりとして定義することで、計算効果の持つプログラムに関する等式変形による検証、モナディック等式変形を
提案した\cite{10.1145/2034574.2034777}。
%モナドトランスフォーマーもしくはtyped store monadについて書く?


\paragraph{Monae}
定理証明支援系Coqでモナディック等式変形を用いた検証を可能にするツールはMonae\cite{monae}である。
Monaeは、モナディック等式変形を行うための等式の集まりであるインターフェイスと、
その健全性を保証するモデルから構成される。
Coqを用いることで、検証の正しさを保証し、MathComp/SSReflectを用いた簡潔な証明が可能になる。
Monaeでは、モナディック等式変形で必要となるインターフェイスの階層を
Hierarchy Builder\cite{cohen:hal-02478907}を用いて実装することで、
複数のモナドの組み合わせや、再利用可能な構造的な証明を可能にする。
%現在monaeでできることについて書く?
\paragraph{構造的でない再帰関数の扱い}
定理証明支援系では無矛盾性の保証のため、停止しない関数を定義できない。
Coqの\coqin{Fixpoint}コマンドでは、引数の持つ整礎な順序関係が構文的に自明な構造的再帰関数については定義できる。
そうでないときは、\coqin{Equation}コマンドなどを用いて引数が整礎な順序関係を持ち減少する値であることの
証明とともに定義する必要がある。
%停止性の証明なしに扱える さらに、構造
%一般再帰が扱える さらに、quick sortなども
さらに非決定性を持つquicksortなどのプログラムでは、引数が減少することを証明することはできず、
dependently-typed assertion\cite{d}などの工夫が必要になる。
%また、停止性の未解決なコラッツ関数などは定義することができない。
%interaction treeを全面的に出す?
\paragraph{CoInductive typeを用いた一般再帰関数の定義}
定理証明支援系で構造的でない再帰関数を扱う他の方法として、余帰納的定義を用いる手法がある。
余帰納的定義は無限長のリストであるストリームなど無限個のコンストラクターを持つデータを定義する際に用いられる。
Coqでは、生産性条件を満たす限り、無限にコンストラクターを適用することができるため、
それを用いて無条件な再帰呼び出しを行い、構造的でない再帰関数を扱うことができる。
それらの関数は、戻り値が余帰納的データとなるが、停止性の証明なしに定義することができる。\\
例えば、停止性の未解決なコラッツ関数を定義することが可能である。
%例えば、\cite{10.1145/3371119}では、coinductive typeを用いて定義されたデータ構造ITreeを用いて
%インタラクションのある一般再帰関数を表現している。\\
Delayモナド\cite{lmcs:2265}を用いるとそういった余再帰的な関数呼び出しを行うプログラムを
モナディックプログラムとして表すことができる。
したがって、本研究では、Delayモナドに対する適切なインターフェイスを定義することで
Monaeを用いた停止性の保証できない一般再帰関数に対する検証を試みた。
%\subsection{interaction treeにおける構造的でない再帰関数の扱い}
\paragraph{Complete elgot monad}
Delayモナドのインターフェイスを定義するにあたって、complete elgot monad\cite{ADAMEK20101306}
を参考にした。
complete elgot monadは、代数的に再帰構造を扱うiteration theory\cite{1993Bloom}に
対応しており、イテレーションと呼ばれる、
各$f : X \rightarrow M (Y + X)$を$f^{\dagger} : X \rightarrow M Y$に対応させるオペレーター
$(\_^{\dagger})$に関する4つの公理を満たすモナドとして定義される。\\
例えば、等式fixpointは次の可換図式で表される。
\xymatrixcolsep{2.5cm}
\[
  \xymatrix{
  X \ar[r]^{f^{\dagger}} \ar[d]^{f} & M A  \\
  M (A + X) \ar[r]^{M[\eta_A, f^{\dagger}]}  & M^2 A \ar[u]^{\mu_A}
  }
\]
この規則は関数fがAの値を返すまで繰り返し計算を行った結果が、$f^{\dagger}$に等しいことを表している。\\
Delayモナドは有限回の計算ステップを無視する同一視を行うことで
complete elgot monadとなることが知られている\cite{10.1007/978-3-319-67729-3_3}。
%\subsection{Typed store monad transformer}

\paragraph{本稿の貢献と構成}
本稿の貢献は以下のようにまとめられる。
\begin{itemize}
  \item Delayモナドの計算的同値性に関する規則を含むインターフェイスを定義し、その健全性を
        形式的に示すことで、一般再帰関数に関する検証をMonaeで行うことを可能にした。
  \item モナドトランスフォーマーを用いた他のモナドとの組み合わせやsetoidライブラリを用いたgeneralized rewrite tactics の利用により
        一般再帰関数に関するMonaeによる検証の有用性を高めた。
\end{itemize}
以下、2節でDelayモナドのインターフェイスの詳細ついて、3節で型付きストアモナドとの組み合わせについて、
4節でsetoidライブラリを用いたgeneralized rewrite tacticsについて、5節で検証の具体例について説明する。
また6節で関連研究について、7節でまとめと課題について論じる。
なお、本研究のコードは以下のulrから確認することができる。\\
\begin{center}
  \url{https://github.com/Ryuji-Kawakami/monae/tree/delaypull}
\end{center}

\section{DelayモナドのMonaeにおける実装}
\subsection{Delayモナドの定義}
%Cofixpointの説明?
Delayモナドは、余帰納的に定義されることで
停止しない関数の計算を表現することが可能である。\\
Delayモナドを構成する関手Delayは、(A:Type) $\rightarrow$ (\text{cofixpoint} of X = A + X )
という型を持つ関数である。
Coqでは、\coqin{CoInductive}コマンドと、各最大不動点への埋め込みを表すコンストラクター\coqin{DNow}, \coqin{DLater}
を用いて定義することができる。
\begin{minted}[bgcolor=bg,numbers=left,xleftmargin=1.5em]{ssr}
CoInductive Delay (A : Type) : Type :=
  |DNow : A -> Delay A
  |DLater : Delay A -> Delay A.
\end{minted}
Delayモナドのreturnオペレーターは、コンストラクタ\coqin{DNow}であり、
bindオペレーターは\coqin{CoFixpoint}コマンドを用いて定義される。
\begin{minted}[bgcolor=bg,numbers=left,xleftmargin=1.5em]{ssr}
Let ret (a:A) := DNow a
CoFixpoint bind (m: Delay A) (f: A -> Delay B ) :=
  match m with
  |DNow a => f a
  |DLater d => DLater (bind d f)
  end.
\end{minted}
さて、Delayモナドに付随するオペレーターとして、繰り返し処理を行うwhileを定義する。
while オペレーターはcomplete elgot monadの$\dagger$オペレーターに相当する。\\
\coqin{CoFixpoint}コマンドを用いて定義され、右埋め込みの値\mintinline{ssr}|inr a|が値a
での繰り返しの継続、左埋め込みの値\mintinline{ssr}|inl b|が値bを戻り値とする繰り返し
の終了を表す。
\begin{minted}[bgcolor=bg,numbers=left,xleftmargin=1.5em]{ssr}
CoFixpoint while {A B} (body: A -> M (B + A)) : A -> M B :=
  fun a => (body a) >>= (fun ab => match ab with
                                   |inr a => DLater (while body a)
                                   |inl b => DNow b end).
\end{minted}
%CoFixpointに関する説明を詳しく?
\subsection{計算的同値性}
さて、ここで等式変形による検証は、Delayモナドを用いて表したプログラムに対しては適さない。
例えば、階乗を計算する関数\coqin{fact}をDelayモナドを用いて定義した場合、\coqin{fact 3}は
明示的な計算ステップ\coqin{DLater}を3つ含むため、\coqin{DNow 6}と一致しないためである。
したがって、計算的な同値性を表す関係が必要である。
そこで、有限個の\coqin{DLater}を除いて等しい場合、またはどちらも\coqin{DLater}が無限個続く場合に計算的に
等しいとみなす関係\coqin{wBisim}を\cite{lmcs:2265}と同様に次のように導入した。\\
まず、計算がある値で停止する性質を\coqin{Terminate}という帰納的な関係で定義する。\\
\mintinline{ssr}|Terminate d a| とは、dが計算の結果、値aを返すことである。
\begin{minted}[bgcolor=bg,numbers=left,xleftmargin=1.5em]{ssr}
Inductive Terminates A : Delay A -> A -> Prop :=
  |TDNow a : Terminates (DNow a) a
  |TDLater d a : Terminates d a -> Terminates (DLater d) a.
\end{minted}
次に、この述語を用いて、\coqin{wBisim}を余帰納的関係として定義する。
つまり、d1,d2が有限個の\coqin{DLater}を除いて等しい場合\coqin{wBTerminate d1 d2 a}が成り立ち、
d1,d2がどちらも\coqin{DLater}が無限個続く場合は、\coqin{wBLater}により
余帰納法を用いることで、\coqin{wBisim d1 d2}を示すことができる。
\begin{minted}[bgcolor=bg,numbers=left,xleftmargin=1.5em]{ssr}
CoInductive wBisim A : Delay A -> Delay A -> Prop :=
  |wBTerminate d1 d2 a :
    Terminates d1 a -> Terminates d2 a -> wBisim d1 d2
  |wBLater d1 d2 : wBisim d1 d2 -> wBisim (DLater d1) (DLater d2).
\end{minted}

\subsection{余帰納法を用いた等式の証明}
%coinductionとcoqのcofixがどう対応しているか
%coqではcofixは、inductionがfixで証明項を作るのと同様に、cofixで証明項を作ることで
%coinductionを実装している。つまり、この例では cofix CIH (d : M A) : Oeq (DLater d) d :=
%から始まる証明項を作ることで、余帰納法を行うことができる。
%さて、cofixの型規則は以下である。(論文引用しながら)
%論文の型規則、ガーデッドコンディションを日本語で書き下す、簡単に
%The predicate Co ensures that, going downwards in M, after a certain number of abstractions and case analysis, there is always a constructor. Once a constructor has been placed, the predicate C1 allows to use f in its recursive arguments. 
%(1)余帰納呼び出しは、CoInductiveのコンストラクタを含まなければならない

%(2)余帰納呼び出しは、CoInductiveのコンストラクタで包まなければならない

%(3)余再帰呼び出しは、CoInductiveのコンストラクタの引数の一番外側で行わなければならない
%つまり、
%その上で、この証明は次のように行った。証明中のcofixで、、、証明の中では、、、からguarded conditionを
%満たしていることがわかる。
%each recursive call in the definition must be protected by at least one constructor, and only by constructors.


ここでは、\coqin{wBisim_DLater}を例にどのように余帰納法の原理を用いて、\coqin{wBisim}に関する性質をCoq上で示したかについて
説明する。

\begin{minted}[bgcolor=bg,numbers=left,xleftmargin=1.5em]{ssr}
  Lemma wBisim_DLater A : forall (d : M A), wBisim (DLater d) d.
  \end{minted}

Coqでは、帰納法で\coqin{fix}オペレーターを用いて証明項を構成するのと同様に、
余帰納法では\coqin{cofix}オペレータを用いて証明項を構成する。
つまり、この例では、\\
\coqin{cofix CIH (d : M A) : Oeq (DLater d) d := |$\cdots$|}
から始まる項を構成すればよい。
さて、\coqin{cofix}に関する型規則は以下である\cite{ddd}。


\begin{prooftree}
  \AxiomC{$\Gamma \vdash B:Set$}
  \AxiomC{$\Gamma, f:B \vdash N:B $}
  \AxiomC{$\mathcal{C}\{ f, N \}$}
  \RightLabel{\text{CoFix}}
  \TrinaryInfC{$\Gamma \vdash \text{CoFix} \, f:B := N:B$}
\end{prooftree}

ただし、CoIを余帰納的に定義された型、PをCoIのパラメーター、$ B := \Pi_{z:Z} (CoI \, P \, (z))$
$\mathcal{C}\{ f, N \}$をfとNに関する条件とする。この条件は構文的ガード制約と呼ばれ、
Nに関して帰納的に定義される。特に次のような性質を満たす\cite{ddd}
\begin{itemize}
  \item Nはコンストラクターを含まなければならない。
  \item 余再帰呼び出しfはコンストラクターの内側で行われ、その外側では、場合わけとラムダ抽象しか行われていない。
\end{itemize}
さて、このことに注意して、次のように証明を行った。\\
cofixタクティックで、余再帰呼び出しである\coqin{apply CIH}は、コンストラクター\coqin{wBLater}
の内側で行われており、その前には\coqin{d}に関する場合わけしか行っていないため、確かに構文的ガード
制約を満たす。

\begin{minted}[bgcolor=bg,numbers=left,xleftmargin=1.5em]{ssr}
  Lemma wBisim_DLater A : forall (d : M A), wBisim (DLater d) d.
  Proof.
  cofix CIH => d.
  case: d => [a|d'].
  - apply : wBTerminate.
    + by apply/TDLater/TDNow.
    + by apply TDNow.
  - apply wBLater.
    by apply CIH.
  Qed.
  \end{minted}

\subsection{Delayモナドのインターフェイス}
以上のことを踏まえて、Delayモナドのインターフェイスを表1,2,3のように定義した。\\
まず、Delayモナドのオペレイターは、繰り返し処理を行う\coqin{while}である。また、計算の等さを表す関係である、
\coqin{wBisim} \, $ \approx $と、それが同値関係であるという規則を導入する。\\
また、\coqin{while}オペレーターに対する6つの規則を定義した。
初めの3つの規則は、Complete elgot monadの定義を参考にした。\\
fixpointEは、\coqin{while}によって、繰り返す処理をすることができることを表す規則である。\\
naturalityEは、while文により行った計算結果を、次の処理に渡す場合、
それを一つのwhile文により記述できることを表す。\\
codiagonalEは、連続して入れ子になったwhile文を一つのwhile文にすることができることを表す。\\
後半の3つの規則は、検証上有用であると考え導入した。\\
bindmwBはbindの引数が同じ計算結果を表しているのならば、fに渡した結果も同じ計算結果になることを表している。\\
bindfwBは同じfとgが同じ計算をするならば、bindで同じ引数を渡した結果も同じ計算結果になることを表す。\\
whilewBは、while文でその都度繰り返す処理が、計算的に等しいならば、
while文全体として等しいことを表す。\\
後半の3つの規則は、whileとbindが同値類を保つことを表している。
したがって、パラメトリックモルフィズムとして登録することで、等式変形に近い証明を可能にした。
詳細については4節で説明する。
これらの規則を用いた等式変形では、モデルにおいて、健全性を示した時のような余帰納法を用いた証明
をする必要がなく、簡潔で直感的な検証が可能になる。

\begin{table}
  \caption{The operator and relation}
  \centering
  \begin{tabular}{|c|}
    \hline
    \coqin{while :  (A -> M(B + A)) -> A -> M B} \\
    \coqin{wBisim :  M A -> M A -> Prop}         \\
    \hline
  \end{tabular}
\end{table}

\begin{table}
  \caption{The laws of the wBisim}
  \centering
  \begin{tabular}{|l|l|}
    \hline
    \coqin{ wBisim_refl}  & \coqin{ a |\scalebox{0.8}{$\approx$}| a}                                                                       \\
    \coqin{ wBisim_sym}   & \coqin{ a |\scalebox{0.8}{$\approx$}| b -> b |\scalebox{0.8}{$\approx$}| a}                                    \\
    \coqin{ wBisim_trans} & \coqin{ a |\scalebox{0.8}{$\approx$}| b -> b |\scalebox{0.8}{$\approx$}| c -> a |\scalebox{0.8}{$\approx$}| c} \\
    \hline
  \end{tabular}
\end{table}

\begin{table}
  \caption{The laws of DelayMonad}
  \centering
  \begin{tabular}{|l|l|}
    \hline
    \coqin{fixpointE }   & \coqin{while f a |\scalebox{0.8}{$\approx$}| (f a) >>= (sum_rect Ret (while f)) }                          \\
    \coqin{naturalityE } & \coqin{while f a >>= g |\scalebox{0.8}{$\approx$}|}                                                        \\
                         & \coqin{while (fun y => (f y) >>=}                                                                          \\
                         & \coqin{  (sum_rect (M |\#| inl o g) (M |\#| inr o Ret))) a}                                                \\
    \coqin{codiagonalE } & \coqin{while ((M |\#| ((sum_rect  idfun inr))) o f ) a |\scalebox{0.8}{$\approx$}|}                        \\
                         & \coqin{  while (while f) a}                                                                                \\
    \coqin{bindmwB}      & \coqin{d1 |\scalebox{0.8}{$\approx$}| d2 -> d1 >>= f |\scalebox{0.8}{$\approx$}| d2>>= f}                  \\
    \coqin{bindfwB}      & \coqin{(forall a, f a |\scalebox{0.8}{$\approx$}| g a) -> d >>= f |\scalebox{0.8}{$\approx$}| d >>= g}     \\
    \coqin{whilewB}      & \coqin{(forall a, f a |\scalebox{0.8}{$\approx$}| g a) -> while f a |\scalebox{0.8}{$\approx$}| while g a} \\
    \hline
  \end{tabular}
\end{table}

\section{型付きストアモナドとの組み合わせ}

Delayモナドのインターフェイスを定義することで、一般再帰関数に対する検証が可能になった。
この手法をより多くの関数へ適用するためには、Delayモナドを他のモナドと組み合わせることにより
複雑な副作用を表現する必要がある。そこで、\cite{practicalformalizationmonadicequational}
において導入された型付きストアモナドとの組み合わせたdelayTypedstoremonadを定義した。

\subsection{型付きストアモナド}
型付きストアモナドは、Coqgen\cite{ValidatingOS}により変換したコードの、
referenceを表すために導入された。\\
型と値のレコードbindingのリストを状態として使うことで、型付きストアを表している。\\
また参照を扱うためのoperatorである\coqin{cnew}, \coqin{cget}, \coqin{cput}を持つ。

\begin{minted}[bgcolor=bg,numbers=left,xleftmargin=1.5em]{ssr}
Record binding :=
  mkbind { bind_type : ml_type; bind_val : coq_type bind_type }.
   \end{minted}
本研究では、型付きストアモナドを、例外モナドトランスフォーマー\coqin{MX}と
状態モナドトランスフォーマー\coqin{MS}の合成で定義された型付きストアモナドトランスフォーマーに
変更し、それぞれのモナドトランスフォーマーがDelayモナドの構造を保つことを示すことで、
delay\_typed\_storemonadを定義した。
\begin{center}

  \coqin{MS (seq binding) option_monad} \ $\Rightarrow$ \
  \coqin{MS (seq binding) (MX unit M)}
\end{center}



\subsection{状態モナドトランスフォーマーとの組み合わせ}
状態モナドトランスフォーマーstateTは次のように定義される。\\
Sが状態、Mが変換前の関手、として\coqin{MS},\coqin{retS},\coqin{bindS}がそれぞれ変換後の関手、return, bindである。

\begin{minted}[bgcolor=bg,numbers=left,xleftmargin=1.5em]{ssr}
Let MS := fun A => S -> M (A * S).
Let retS := fun A => curry Ret.
Let bindS m f := fun s => m s >>= uncurry f.
Let liftS m := fun s => m >>= (fun x => Ret (x, s)).
        \end{minted}
さて、stateTにより状態モナドと組み合わせるためには、delayモナドがstateTで変換後、delayモナドであることを示す
必要がある。そこで、\cite{PIROG2013309}を参考に次のように\coqin{whileDS}と\coqin{wBisimDS}を定義した。
関数\coqin{dist1}は、分配法則を表し、型を合わせるために定義した。
\iffalse
  そこで、次の定理を用いた。\\

  %monad for behaviorの定理をかく
  %ここもなくてもよい。参考にした論文のい挙げる
  随伴にcurryとuncurryを

  特に状態モナドが随伴$(- \otimes A) \vdash (A \rightarrow - )$から導出されるため次のように
  while operatorとwBisimを定義した。関数dist1は、分配法則を表し、型を合わせるために定義した。
\fi
\begin{minted}[bgcolor=bg,numbers=left,xleftmargin=1.5em]{ssr}
M:delayMonad
Let DS := MS M
Let whileDS (body : X -> DS (Y + X)) :=
  curry (while (M # dist1 o uncurry body)).

Let wBisimDS (ds1 ds2 : DS A) : Prop :=
  forall s : S, wBisim (ds1 s) (ds2 s).
\end{minted}

\subsubsection{例外モナドトランスフォーマーとの組み合わせ}
例外モナドトランスフォーマーexceptTは次のように定義される。\\
Zが例外の集合、Mが変換前の関手、として\coqin{MX},\coqin{retX},\coqin{bindX}がそれぞれ変換後の関手、return, bindである。

\begin{minted}[bgcolor=bg,numbers=left,xleftmargin=1.5em]{ssr}
Let MX := fun X => M (Z + X).
Let retX : idfun --> MX := fun X x => Ret (inr x).
Let bindX t f :=
  t >>= fun c => match c with inl z => Ret (inl z) | inr x => f x end.
        \end{minted}
stateTの場合と同様に、delayモナドがexceptTで変換後delayモナドであることを示す必要がある。
そこで、次のように定義した。
関数\coqin{DEA}を合成することでにより、エラーが発生した際\coqin{inl (inl u)}を返すことで、繰り返しを終了する。

\begin{minted}[bgcolor=bg,numbers=left,xleftmargin=1.5em]{ssr}
M: delayMonad
Let DE := MX unit M.
Let whileDE (body : A -> DE (B + A)) : DE B := while (DEA o body)
Let wBisimDE (d1 d2 : DE A) := wBisim d1 d2.
        \end{minted}

\subsection{定義できる関数の例}
\coqin{M : delay_typed_storemonad}を用いることで、参照に関する
操作とwhileを用いた繰り返しを含むプログラムを表現可能である。\\
例えば、参照とwhile文を用いて階乗を計算するプログラム\coqin{factdts}を以下のように定義できる。


\begin{minted}[bgcolor=bg,numbers=left,xleftmargin=1.5em]{ssr}
Let factdts_aux_body r n : M (unit + nat) :=
  do v <- cget r;
     match n with
     |O => do _ <- cput r v; Ret (inl tt)
     |S m => do _ <- cput r (n * v); Ret (inr m)
     end.
Let factdts_aux n r := while (factdts_aux_body r) n.
Let factdts n := do r <- cnew ml_int 1;
                 do _ <- factdts_aux n r ;
                 do v <- cget r; Ret v.
            \end{minted}

\section{一般書き換え}
Setoidライブラリは、ユーザーが独自に定義したパラメトリック同値関係に対するrewrite tacticの使用を
可能にする。
一般再帰関数に対する検証では、等式変形による検証ではなく、計算的な同値性であるwBisimに
関する等価性変形を行う必要がある。\\
したがって、wBisimをパラメトリック同値関係としてインスタンス化した。
\begin{minted}[escapeinside=||,bgcolor=bg,numbers=left,xleftmargin=1.5em]{ssr}
Add Parametric Relation A : (M A) (@wBisim M A)
  reflexivity proved by (@wBisim_refl M A)
  symmetry proved by (@wBisim_sym M A)
  transitivity proved by (@wBisim_trans M A) as wBisim_rel.
        \end{minted}
インターフェイスの規則であるbindmwBとbindfwB、whilewBがそれぞれ
関数\coqin{bind}, \coqin{while}が同値関係\coqin{wBisim}を保つことを表している。
したがって、\coqin{bind},\coqin{while}を Parametric morphismとしてインスタンス化した。
定義の中に現れる\\
\coqin{(pointwise_relation A (@wBisim M B))}とは関数の関係であって、
任意の値に対して関数が同じ同値類の値を返すことを表す。
これにより、図1,2のような書き換えが可能になる。
さらに、\coqin{setoid_rewrite}を用いることで、束縛変数
を含む項に関する書き換えが可能である。このこととついては、5.3節で具体的に説明する。

\begin{minted}[escapeinside=||,bgcolor=bg,numbers=left,xleftmargin=1.5em]{ssr}
Add Parametric Morphism A B : bind
  with signature (@wBisim M A) ==>
    (pointwise_relation A (@wBisim M B)) ==> (@wBisim M B) as bindmor.

  Add Parametric Morphism A B : while
  with signature (pointwise_relation A (@wBisim M (B + A))) ==>
    @eq A ==> (@wBisim M B ) as whilemor.
  \end{minted}
これにより、Monaeを用いた他の検証と同様なrewrite tacticsを中心とした証明が可能になった。
\begin{figure}[h]
  \centering
  \begin{minipage}{0.38\textwidth}
    \begin{minted}[escapeinside=||]{ssr}
 H1 : d1 |$\approx$| d2
 H2 : forall a, f |$\approx$| g
| \xdotfill{1pt}[black] |
 d1 >>= f |$\approx$ \, $\cdots$|
  \end{minted}
  \end{minipage}
  \begin{minipage}{0.2\textwidth}
    \begin{center}
      \scalebox{1.1}{$\xRightarrow[\coqin{rewrite H1 H2.}]{}$}
    \end{center}
  \end{minipage}
  \begin{minipage}{0.38\textwidth}
    \begin{minted}[escapeinside=||]{ssr}
H1 : d1 |$\approx$| d2
H2 : forall a, f a |$\approx$| g a
|\xdotfill{1pt}[black] |
d2 >>= g |$\approx$ \, $\cdots$|
  \end{minted}
  \end{minipage}
  \caption{rewrite with bind}
\end{figure}

\begin{figure}[h]
  \centering
  \begin{minipage}{0.38\textwidth}
    \begin{minted}[escapeinside=||]{ssr}
 H : forall a, f a |$\approx$| g a.
| \xdotfill{1pt}[black] |
 while f b |$\approx$ \, $\cdots$|
  \end{minted}
  \end{minipage}
  \begin{minipage}{0.2\textwidth}
    \begin{center}
      \scalebox{1.1}{$\xRightarrow[\coqin{rewrite H.  }]{}$}
    \end{center}
  \end{minipage}
  \begin{minipage}{0.38\textwidth}
    \begin{minted}[escapeinside=||]{ssr}
H : forall a, f a |$\approx$| g a.
|\xdotfill{1pt}[black] |
while g b |$\approx$ \, $\cdots$|
  \end{minted}
  \end{minipage}
  \caption{rewrite with while}
\end{figure}

\section{検証の具体例}
ここでは、Delayモナドのインターフェイスを用いた検証の具体例を紹介する。

\subsection{Monadの満たす等式}
モナドは、bindとreturnを持ち、表4の等式を満たす型クラスとして特徴づけられる。
\begin{table}[b]
  \caption{The laws of the monad}
  \centering
  \begin{tabular}{|l|l|}
    \hline
    \coqin{bindretf} & \coqin{Ret a >>= f = f a}                            \\
    \coqin{bindmret} & \coqin{m >>= Ret = Ret}                              \\
    \coqin{bindA}    & \coqin{(m >>= f) >>= g = m >>= (fun x => f x >>= g)} \\
    \hline
  \end{tabular}
\end{table}

Monaeでは、各Monadをインスタンス化した際にこれらの等式と関連する補題を用いることができるようになる。
特にここでは、表4の等式を用いて検証する。

\subsection{マッカーシーの91関数に対する検証}
マッカーシーの91関数mc91は、再帰的に定義される関数であり、$m \leq 101$の時、
必ず91を返すという性質をもつ関数である。

\begin{minted}[bgcolor=bg,numbers=left,xleftmargin=1.5em]{ocaml}
let rec mc91 m = if 100 < m then m - 10 else mc91 (mc91 (m+11))
   \end{minted}

$m \leq 100$の時、11足した値で二重に再帰するため、停止性がCoqでは直ちに
判定できない関数である。\\
したがって、mc91関数をdelayモナドを用いて表し、自然数$m \leq 100$について
91が戻り値となることを示した。\\
まず、\coqin{while}オペレーターを用いてmc91関数を用いて表す。残っている再帰の深さn、mが計算している値である。

\begin{minted}[bgcolor=bg,numbers=left,xleftmargin=1.5em]{ssr}
Let mc91_body nm :=
    match nm with (n, m) =>
    if n==0 then ret (inl m)
            else if m > 100
                 then ret (inr(n.-1, m - 10))
                 else ret (inr(n.+1, m + 11))
    end.
Let mc91 n m := while mc91_body (n.+1, m).
    \end{minted}

mc91関数が91を返すことを示す際、本質的な性質は、補題\coqin{mc91succE}である。
この補題と、\coqin{mc91 n 101 |$\approx$| Ret 91}であることと、$ k =  90 - m $に関する帰納法により従う。

\begin{minted}[bgcolor=bg,numbers=left,xleftmargin=1.5em, escapeinside=||]{ssr}
Lemma mc91succE n m : 90 <= m < 101 -> mc91 n m |\scalebox{0.8}{$\approx$}| mc91 n (m.+1).
\end{minted}

さて、この補題を図1のように示した。

\begin{figure}[H]
  \centering
  \begin{minted}[escapeinside=||,mathescape = true, frame = single]{ssr}
mc91 n m
 |$\llbracket$| rewrite /mc91. (*definition of mc91*) |$\rrbracket$|
 | \scalebox{0.8}{$\approx$} | while mc91_body (n.+1, m)
 |$\llbracket$| rewrite fixpointE. |$\rrbracket$|
 | \scalebox{0.8}{$\approx$} | (if 100 < m then Ret (inr (n, m - 10))
              else Ret (inr (n.+2, m + 11))) >>=
                   sum_rect Ret (while mc91_body)
 |$\llbracket$| rewrite ifN //. (* $ m \leq 100$ *)|$\rrbracket$|
  |\scalebox{0.8}{$\approx$}| Ret (inr (n.+2, m + 11)) >>= sum_rect Ret (while mc91_body)
 |$\llbracket$| rewrite bindretf /= fixpointE /=. |$\rrbracket$|
  |\scalebox{0.8}{$\approx$}| while mc91_body (n.+2, m + 1)
 |$\llbracket$| rewrite bindretf fixpointE /=. |$\rrbracket$|
  |\scalebox{0.8}{$\approx$}| (if 100 < m + 11
      then Ret (inr (n.+1, m + 11 - 10))
      else Ret (inr (n.+3, m + 11 + 11))) >>=
          sum_rect Ret (while mc91_body)
  |$\llbracket$| rewrite ltn_add2r ifT //. (* $90 \leq m \implies 100 < m + 11 $ *) |$\rrbracket$|
  Ret (inr (n.+1, m + 11 - 10)) >>= sum_rect Ret (while mc91_body)
 |$\llbracket$| rewrite  bindretf fixpointE /= fixpointE. |$\rrbracket$|
  |\scalebox{0.8}{$\approx$}|  while mc91_body (n.+1, m + 11 - 10) = mc91 n (m+1)
    \end{minted}
  \caption{proof of mc91 }
\end{figure}

\subsection{状態を用いたfactorial}
3.3節で定義した\coqin{factdts}について、実際に階乗を計算していること
を検証する。つまり次\coqin{factn}と計算として一致することを示した。

\begin{minted}[bgcolor=bg,numbers=left,xleftmargin=1.5em]{ssr}
Fixpoint fact n := match n with |O => 1 |m.+1 => n * fact m end.
Let factn n := do r <- cnew ml_int (fact n);
               do v <- cget r; @ret M _ v.
\end{minted}

ここでは、\coqin{factdts}の\coqin{while}を用いた部分\coqin{fact_aux}が\coqin{fact}を用いた形に書き換えられることについて
取り上げる。この書き換えと、型付きストアモナドの等式を用いることで、\coqin{factn}に一致することが従う。
証明はnに関する帰納法により図4,5のように行った。\\
\coqin{under}タクティックを用いることで、bindに関する束縛変数を含む項に関する等式
を行っている。

\begin{figure}[H]
  \centering
  \begin{minted}[escapeinside=~~, frame = single]{ssr}
while (factdts_aux_body r) 0
~$\llbracket$~ rewrite fixpointE/= !bindA. ~$\rrbracket$~
~ \scalebox{0.8}{$\approx$} ~ cget r >>= (fun s => (cput r s >> Ret (inl tt)) >>=
                     sum_rect Ret (while (factdts_aux_body r)))
~$\llbracket$~ under eq_bind => s. (* rewrite under binder *) ~$\rrbracket$~
'Under[ (cput r s >> Ret (inl tt)) >>=
         sum_rect Ret (while (factdts_aux_body r)) ]
~$\llbracket$~ rewrite bindA bindretf/= -{1}(mul1n s). ~$\rrbracket$~ 
'Under[ cput r (1 * s) >> Ret tt ]
~$\llbracket$~ over. ~$\rrbracket$~
cget r >>= (fun s => cput r (1 * s) >> Ret tt)
= cget r >>= (fun s => cput r (fact 0 * s) >> Ret tt)
    \end{minted}
  \caption{proof of factn if n =  0 }
\end{figure}

n = n' + 1の時の証明の際、\coqin{setoid_rewrite IH}の部分では、
束縛変数を含む項に対する\coqin{wBisim}に関する等価性変形を行っている。これは、
第3節で説明したように、bindをパラメトリックモルフィズムとして、インスタンス化
しているため、行うことができる。

\begin{figure}[H]
  \centering
  \begin{minted}[escapeinside=~~, frame = single]{ssr}
while (factdts_aux_body r) n'.+1
~$\llbracket$~ rewrite fixpointE/= !bindA. ~$\rrbracket$~
~ \scalebox{0.8}{$\approx$} ~cget r >>= (fun s => (cput r (n'.+1 * s) >> Ret (inr n')) >>=
                     sum_rect Ret (while (factdts_aux_body r)))
~$\llbracket$~ under eq_bind => s (* rewrite under binder *) ~$\rrbracket$~
'Under[ (cput r (n'.+1 * s) >> Ret (inr n')) >>=
     sum_rect Ret (while (factdts_aux_body r)) ]
~$\llbracket$~ rewrite bindA bindretf/=. ~$\rrbracket$~
'Under[ cput r (n'.+1 * s) >> while (factdts_aux_body r) n' ]
~$\llbracket$~ over. (* over *) ~$\rrbracket$~
cget r >>= (fun s => cput r (n'.+1 * s) >>
           while (factdts_aux_body r) n')
~$\llbracket$~ setoid_rewrite  IH (*rewrite using induction hypothese IH*)~$\rrbracket$~
~ \scalebox{0.8}{$\approx$} ~ cget r >>= (fun s => cput r (n'.+1 * s) >>
           (cget r >>= (fun s0 => cput r (fact n' * s0) >> Ret tt)))
~$\llbracket$~ under eq_bind => s (* rewrite under binder *) ~$\rrbracket$~
'Under[ cput r (n'.+1 * s) >>
       (cget r >>= (fun s0 => cput r (fact n' * s0) >> Ret tt)) ]
~$\llbracket$~ rewrite cputget -bindA cputput. (*laws for cput and cget *) ~$\rrbracket$~
  'Under[ cput r (fact n' * (n'.+1 * x)) >> Ret tt ]
~$\llbracket$~ over. (* over *) ~$\rrbracket$~
cget r >>= (fun s => cput r (n'.+1 * fact n' * s) >> Ret tt)
= cget r >>= (fun s => cput r (fact n'.+1 * s) >> Ret tt)
       \end{minted}
  \caption{proof of factn if n =  n' + 1 }
\end{figure}
\iffalse
  その後、その書き換えとtypedstoremonadの書き換え規則を用いることで、factnに一致することを示した。

  \begin{minted}[escapeinside=||]{ssr}
  Lemma factE n : factdts n |\scalebox{0.8}{$\approx$}| factn n.

  cnew ml_int 1 >>= (fun r : loc ml_int => factdts_aux n r >> (cget r >>= [eta Ret]))
  |《 rewrite under cnew mlint 1 >>= by bindfwB 》|

  factdts_aux n a >> (cget a >>= [eta Ret])
  |《 fixpointE 》|

  |《 rewrite under cnew mlint 1 >>= by bindfwB》|
  factdts_aux n a >> (cget a >>= [eta Ret])
  |《 fixpointE 》|

  cget a >>=
  (fun x : coq_type N ml_int => (cput a (fact n * x) >> Ret tt) >> (cget a >>= [eta Ret]))
  |《 rewrite under cget a >>= by eqbind》|
  (cput a (fact n * x) >> Ret tt) >> (cget a >>= [eta Ret])
  |《 fixpointE 》|
  cput a (fact n * x) >> (cget a >>= [eta Ret])
  |《 over 》|
  cnew ml_int 1 >>=
  (fun a : loc ml_int =>
   cget a >>=
   (fun x : coq_type N ml_int => cput a (fact n * x) >> (cget a >>= [eta Ret])))
   |《 fixpointE 》|
   cnew ml_int (fact n) >>= (fun r : loc ml_int => cget r >>= [eta Ret])

\end{minted}
\fi

\section{関連研究}
\subsection{一般再帰関数に対するCoq上での検証}
\cite{10.1145/3371119}は、coinductive typeを用いて定義されたデータ構造ITreeを用いて
インタラクションのある一般再帰関数を表現している。この手法では、それぞれの副作用をイベントとしてITreeに埋め込み、
ITreeの間の弱双模倣性を満たす等価関係を定義することでプログラムの検証を可能にしている。
またそれに関連して\cite{10.1145/3372885.3373813}では、
ライブラリpakoを拡張したgpacoを用いることで弱双模倣性に関する等式論理をCoqで実装している。
% Dijkstra Monads Forever: Termination
%free monadを使った手法

\subsection{Elgotモナドに関する理論}
Complete elgot monadに関する理論的研究としては、SergeyとPirógらがが、Compete elogit monadの構造が
いくつかのモナドトランスフォーマーにより保たれることを示している\cite{GONCHAROV2015183}, \cite{PIROG2014273}。
Simpsonらは、はiteration theoryの公理が完全であることを示している\cite{cacmfpo}。

\section{まとめと今後の課題}
本研究により、Delayモナドを通じてMonaeで一般再帰関数を扱うことができるようになった。
等価関係に基づく変形による一般再帰関数の検証の利点はとしては、まずcoinductionによる証明をする必要がないことである。
Coqではcoinductive typeを用いて直接一般再帰関数を表すことが可能ではあるが、検証には余帰納法が不可欠となる。
その際、Coqではガード制約というコンストラクタと余再帰呼び出しに関する構文的制約を満たす必要があり、証明が難しくなる。
一方、等価関係に基づく変形では、そのような難しさは生じず、直感的で簡潔な証明が可能になる。
次に、他の副作用との組み合わせが容易であることもわかった。
本研究で取り上げたように、再帰構造をモナドを用いて表現することで、モナドトランスフォーマーによる
型付きストアモナドとの組み合わせを容易に行うことが可能である。
さらに、インターフェイスにオペレーターや規則を追加することで他の副作用とも組み合わせることができると期待している。
また、これまでのMonaeでの検証では等式変形による検証のみ扱っていたが、
Setoidライブラリを用いることで、等価関係に基づく変形による検証もこれまでと同様に
rewrite タクティクスを中心とした証明により行うことが可能であることがわかった。\\
今後の課題としては以下である。

\begin{enumerate}
  \item 他のモナドとの組み合わせ\\
        今回の研究では、モナドトランスフォーマーによるモナドの組み合わせしか取り扱っていないが、
        モナディック等式変形では、インターフェイスに異なるモナドのオペレーターとその間の等式を
        追加することで、複数の副作用を持つプログラムに関する検証を可能にしている。
        特にモナドトランスフォーマーによる組み合わせが難しい確率モナドや非決定性モナドなどとの組み合わせが
        可能かどうかが課題である。
  \item 一般的な双模倣性に関するMonaeでの検証\\
        今回の研究で用いた計算的同値性は、Delay A上の弱双模倣性を満たす関係である。
        一方で一般的な双模倣性に関する等価性変形を行うことで、並行プログラムやStreamなどの
        無限の長さも持ちうるデータを扱うプログラムに関する検証が期待できる。そのような
        一般的な双模倣性に関する性質を持つモナドのインターフェイスをどのように定義するか、
        またその健全性をどのように示すかが課題である。
  \item プログラムの部分正当性の検証\\
        現在、プログラムの実行後の状態について検証するには、factauxEの証明で行ったように
        整礎関係を見つけ出し、その関係に関する帰納法を行う必要がある。
        つまり、実質的にプログラムの停止性を示す必要があり、ホーア論理で行うようなループ不変条件を用いた
        検証を行うことができない。したがって、ループ不変条件を用いた部分正当性の検証を
        どのように等式変形の枠組みで導入するかが課題である。
\end{enumerate}

\bibliographystyle{alpha}
\bibliography{reference.bib}
\end{document}